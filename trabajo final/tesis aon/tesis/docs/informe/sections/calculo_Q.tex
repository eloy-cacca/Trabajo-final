%% ----------------------------------------------------------------
%% SETTINGS
%% ----------------------------------------------------------------
\documentclass[../main.tex]{subfiles}

%% ----------------------------------------------------------------
%% BEGIN
%% ----------------------------------------------------------------
\begin{document}

%% ----------------------------------------------------------------
%% DOCUMENT
%% ----------------------------------------------------------------
\section{Calculo Q}
Para optimizar el funcionamiento se calculo los valores maximos que puede tomar una variable tipo int32 para distintos valores de Q, usando la cuenta:
\[
    \text{Max} = \frac{2^{31 - Q}}{2^Q}
\]
obteniendo la siguiente tabla:

\begin{table}[!hbtp]
    \centering
    \begin{tabular}{@{}cc@{}}
    \toprule
    Q  & Max  \\ \midrule
    $15$ & $2$    \\
    $14$ & $8$    \\
    $13$ & $32$   \\
    $12$ & $128$  \\
    $11$ & $512$  \\
    $10$ & $2048$ \\ \bottomrule
    \end{tabular}
\end{table}

Para el peor caso voy a simular una señal continua.

\subsection{Limitante 1: signal\_processing RMS2 buffer}
En este caso se hace un buffer de 128 y se suma el RMS2 en cada uno usando una señal int64. Luego el valor maximo antes del overflow es:
\[
    \text{Max} = \frac{\sqrt{2^{63 - Q} / 128}}{2^Q}
\]
obteniendo la siguienta tabla:

\begin{table}[!hbtp]
    \centering
    \begin{tabular}{@{}cc@{}}
    \toprule
    Q  & Max  \\ \midrule
    $15$ & $45.25$ \\
    $14$ & $128$   \\
    $13$ & $362$   \\
    $12$ & $1024$  \\
    $11$ & $2896$  \\
    $10$ & $8192$  \\ \bottomrule
    \end{tabular}
\end{table}

\subsection{Limitante 2: index\_accum RMS2 buffer}
Este acumula los valores del anterior, hasta 128 veces. Luego el maximo es:
\[
    \text{Max} = \frac{\sqrt{2^{63 - Q} / (128 \cdot 128)}}{2^Q}
\]
obteniendo la siguienta tabla:

\begin{table}[!hbtp]
    \centering
    \begin{tabular}{@{}cc@{}}
    \toprule
    Q  & Max  \\ \midrule
    $15$ & $4$ \\
    $14$ & $11.3$   \\
    $13$ & $32$   \\
    $12$ & $90$  \\
    $11$ & $256$  \\
    $10$ & $724$  \\ \bottomrule
    \end{tabular}
\end{table}

\subsection{Resultado}
Se elige entonces utilizar Q12. De esta forma se deja un margen y se obtiene una precisión buena de todas formas, con un escalón de $2^{-12} \approx 0.0002$

%% ----------------------------------------------------------------
%% END
%% ----------------------------------------------------------------
\end{document}